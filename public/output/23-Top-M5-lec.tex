% Options for packages loaded elsewhere
\PassOptionsToPackage{unicode}{hyperref}
\PassOptionsToPackage{hyphens}{url}
%
\documentclass[
  ignorenonframetext,
  t,handout]{beamer}
\usepackage{pgfpages}
\setbeamertemplate{caption}[numbered]
\setbeamertemplate{caption label separator}{: }
\setbeamercolor{caption name}{fg=normal text.fg}
\beamertemplatenavigationsymbolsempty
% Prevent slide breaks in the middle of a paragraph
\widowpenalties 1 10000
\raggedbottom
\setbeamertemplate{part page}{
  \centering
  \begin{beamercolorbox}[sep=16pt,center]{part title}
    \usebeamerfont{part title}\insertpart\par
  \end{beamercolorbox}
}
\setbeamertemplate{section page}{
  \centering
  \begin{beamercolorbox}[sep=12pt,center]{part title}
    \usebeamerfont{section title}\insertsection\par
  \end{beamercolorbox}
}
\setbeamertemplate{subsection page}{
  \centering
  \begin{beamercolorbox}[sep=8pt,center]{part title}
    \usebeamerfont{subsection title}\insertsubsection\par
  \end{beamercolorbox}
}
\AtBeginPart{
  \frame{\partpage}
}
\AtBeginSection{
  \ifbibliography
  \else
    \frame{\sectionpage}
  \fi
}
\AtBeginSubsection{
  \frame{\subsectionpage}
}
\usepackage{amsmath,amssymb}
\usepackage{iftex}
\ifPDFTeX
  \usepackage[T1]{fontenc}
  \usepackage[utf8]{inputenc}
  \usepackage{textcomp} % provide euro and other symbols
\else % if luatex or xetex
  \usepackage{unicode-math} % this also loads fontspec
  \defaultfontfeatures{Scale=MatchLowercase}
  \defaultfontfeatures[\rmfamily]{Ligatures=TeX,Scale=1}
\fi
\usepackage{lmodern}
\usetheme[]{CambridgeUS}
\usecolortheme{dolphin}
\ifPDFTeX\else
  % xetex/luatex font selection
\fi
% Use upquote if available, for straight quotes in verbatim environments
\IfFileExists{upquote.sty}{\usepackage{upquote}}{}
\IfFileExists{microtype.sty}{% use microtype if available
  \usepackage[]{microtype}
  \UseMicrotypeSet[protrusion]{basicmath} % disable protrusion for tt fonts
}{}
\makeatletter
\@ifundefined{KOMAClassName}{% if non-KOMA class
  \IfFileExists{parskip.sty}{%
    \usepackage{parskip}
  }{% else
    \setlength{\parindent}{0pt}
    \setlength{\parskip}{6pt plus 2pt minus 1pt}}
}{% if KOMA class
  \KOMAoptions{parskip=half}}
\makeatother
\usepackage{xcolor}
\newif\ifbibliography
\usepackage{longtable,booktabs,array}
\usepackage{calc} % for calculating minipage widths
\usepackage{caption}
% Make caption package work with longtable
\makeatletter
\def\fnum@table{\tablename~\thetable}
\makeatother
\usepackage{graphicx}
\makeatletter
\def\maxwidth{\ifdim\Gin@nat@width>\linewidth\linewidth\else\Gin@nat@width\fi}
\def\maxheight{\ifdim\Gin@nat@height>\textheight\textheight\else\Gin@nat@height\fi}
\makeatother
% Scale images if necessary, so that they will not overflow the page
% margins by default, and it is still possible to overwrite the defaults
% using explicit options in \includegraphics[width, height, ...]{}
\setkeys{Gin}{width=\maxwidth,height=\maxheight,keepaspectratio}
% Set default figure placement to htbp
\makeatletter
\def\fps@figure{htbp}
\makeatother
\setlength{\emergencystretch}{3em} % prevent overfull lines
\providecommand{\tightlist}{%
  \setlength{\itemsep}{0pt}\setlength{\parskip}{0pt}}
\setcounter{secnumdepth}{5}
% definitions for citeproc citations
\NewDocumentCommand\citeproctext{}{}
\NewDocumentCommand\citeproc{mm}{%
  \begingroup\def\citeproctext{#2}\cite{#1}\endgroup}
\makeatletter
 % allow citations to break across lines
 \let\@cite@ofmt\@firstofone
 % avoid brackets around text for \cite:
 \def\@biblabel#1{}
 \def\@cite#1#2{{#1\if@tempswa , #2\fi}}
\makeatother
\newlength{\cslhangindent}
\setlength{\cslhangindent}{1.5em}
\newlength{\csllabelwidth}
\setlength{\csllabelwidth}{3em}
\newenvironment{CSLReferences}[2] % #1 hanging-indent, #2 entry-spacing
 {\begin{list}{}{%
  \setlength{\itemindent}{0pt}
  \setlength{\leftmargin}{0pt}
  \setlength{\parsep}{0pt}
  % turn on hanging indent if param 1 is 1
  \ifodd #1
   \setlength{\leftmargin}{\cslhangindent}
   \setlength{\itemindent}{-1\cslhangindent}
  \fi
  % set entry spacing
  \setlength{\itemsep}{#2\baselineskip}}}
 {\end{list}}
\usepackage{calc}
\newcommand{\CSLBlock}[1]{\hfill\break\parbox[t]{\linewidth}{\strut\ignorespaces#1\strut}}
\newcommand{\CSLLeftMargin}[1]{\parbox[t]{\csllabelwidth}{\strut#1\strut}}
\newcommand{\CSLRightInline}[1]{\parbox[t]{\linewidth - \csllabelwidth}{\strut#1\strut}}
\newcommand{\CSLIndent}[1]{\hspace{\cslhangindent}#1}
\graphicspath{{../../../../static/}}
\usepackage{color}
\usepackage{marvosym}
\usepackage{amsmath}
\usepackage{amsthm}
\usepackage{amsfonts}
\usepackage{array}
\usepackage{booktabs}
\usepackage{verbatim}
\usepackage[english]{varioref}
\usepackage{natbib}
\bibliographystyle{elsarticle-harv}
\usepackage{hyperref}
\usepackage{actuarialangle}
\usepackage{pgfpages}
\pgfdeclareimage[height=1cm]{university-logo}{../../../../static/img/PRIMARY_A_Vertical_Housed_RGB.png}
\pgfdeclareimage[height=2.5cm]{university-logo2}{../../../../static/img/PRIMARY_A_Vertical_Housed_RGB.png}
\logo{\raisebox{-3ex}[0pt]{\pgfuseimage{university-logo}}}
\AtBeginSection[]{     \begin{frame}    \tableofcontents[sectionstyle=show/shaded,subsectionstyle=hide/hide/hide]     \end{frame}  \addtocounter{framenumber}{-1}}
\AtBeginSubsection[]{     \begin{frame}    \tableofcontents[sectionstyle=show/hide,subsectionstyle=show/shaded/hide]      \end{frame}  \addtocounter{framenumber}{-1}}
\definecolor{DolphinBlue}{RGB}{51,44,159}
\setbeamerfont{section in toc}{size=\normalsize}
\setbeamerfont{subsection in toc}{size=\normalsize}
\pretocmd{\tableofcontents}{\setlength{\parskip}{.2em}}{}{}
\setbeamertemplate{footline}{\hspace*{.4em} \raisebox{1.5ex}[0pt]{\textcolor{DolphinBlue}{\insertframenumber/\inserttotalframenumber}}}
\newcommand{\adv}{$\maltese$}
\ifLuaTeX
  \usepackage{selnolig}  % disable illegal ligatures
\fi
\IfFileExists{bookmark.sty}{\usepackage{bookmark}}{\usepackage{hyperref}}
\IfFileExists{xurl.sty}{\usepackage{xurl}}{} % add URL line breaks if available
\urlstyle{same}
\hypersetup{
  pdftitle={M5 Reserving Combination},
  pdfauthor={Professor Benjamin Avanzi},
  hidelinks,
  pdfcreator={LaTeX via pandoc}}

\title{M5 Reserving Combination}
\subtitle{Topics in Insurance, Risk, and Finance \footnote<.->{References:
  Chapter 5 of Taylor (2000) \textbar{} \(\; \rightarrow\)
  \href{https://gim-am3.netlify.app/output/23-Top-M5-lec.pdf}{\textcolor{blue}{\underline{latest slides}}}}}
\author{Professor Benjamin Avanzi}
\date{01 November 2023}
\institute{\includegraphics[width=1.2in,height=\textheight]{../../../../static/img/PRIMARY_A_Vertical_Housed_RGB.png}}

\begin{document}
\frame{\titlepage}

\begin{frame}[allowframebreaks]
  \tableofcontents[hideallsubsections]
\end{frame}
\section{Background}\label{background}

\subsection{Summary}\label{summary}

\begin{frame}{Summary}
We have seen covered a number of reserving methods:

\begin{enumerate}
\tightlist
\item
  unadjusted chain ladder on paid losses
\item
  inflation adjusted chain ladder on paid losses
\item
  unadjusted chain ladder on incurred losses
\item
  inflation adjusted chain ladder on incurred losses
\item
  separation
\item
  payments per claim incurred (PPCI)
\end{enumerate}

The IBNR claims (counts) were estimated according to a number of
methods, too:

\begin{itemize}
\tightlist
\item
  exposure
\item
  normaliser
\item
  chain ladder
\end{itemize}
\end{frame}

\subsection{Motivation}\label{motivation}

\begin{frame}{Motivation}
\begin{itemize}
\tightlist
\item
  All those methods are producing outstanding liability \$ estimates in
  nominal terms.
\item
  There are a number of differences:

  \begin{itemize}
  \tightlist
  \item
    whether inflation is accounted for explicitly (2, 4, 5, 6) or not
    (1, 3)
  \item
    whether inflation is assumed for past data (2, 4), or a result of
    the method (1, 3, 5, 6)
  \item
    whether paid (1, 2, 6) or incurred (3, 4, 5) losses are used
  \item
    whether aggregate amounts (1, 2, 3, 4) or amounts per claim (5, 6)
    are used. The latter require an estimate of IBNR counts.
  \end{itemize}
\item
  All methods have strengths and weaknesses.
\end{itemize}

There is obviously a good mix of assumptions and approaches. Which one
should we choose? Or should we combine them?
\end{frame}

\subsection{Criteria for choice}\label{criteria-for-choice}

\begin{frame}{Criteria for choice}
Decision factors include:

\begin{itemize}
\tightlist
\item
  analytical properties of the various models

  \begin{itemize}
  \tightlist
  \item
    e.g., we know there has been a change in the past, that affected the
    development of claims. Can the method allow for that?
  \end{itemize}
\item
  average claim sizes for various periods of origin

  \begin{itemize}
  \tightlist
  \item
    e.g., do those average claims make sense, given our (also
    qualitative) knowledge of the claims development processes??
  \item
    e.g., do trends in average claim size agree with our beliefs around
    claims inflation and superimposed inflation?
  \end{itemize}
\item
  relation of forecasts of liability to case estimates

  \begin{itemize}
  \tightlist
  \item
    e.g., they are not directly comparable, but the evolution of the
    ratio should be smooth enough.
  \end{itemize}
\end{itemize}
\end{frame}

\section{Combining the results of the different
models}\label{combining-the-results-of-the-different-models}

\subsection{Idea}\label{idea}

\begin{frame}{Idea}
\begin{itemize}
\tightlist
\item
  Let \(\hat{P}_h^*(i,j)\) be the estimate of outstanding liability of
  model \(h=1,2\ldots\), at end of development period \(j\) and for
  period of origin \(i\).
\item
  Assume the we want to combine such estimates (at the end of experience
  period \(k\) ) as follows:
  \[\overline{\hat{P}^*}(i,k) = \sum_h w_h(i) \overline{\hat{P}^*_h}(i,k),\]
  where \(\overline{\hat{P}^*_h}(i,k) = \hat{P}_h^*(i,k-i)\), \(w_h(i)\)
  are weights allocated to to model \(h\), \[\sum_h w_h(i)=1.\]
\item
  Note that in general those weights depend on \(i\) as well; different
  reserving models will typically perform better for different levels of
  maturity (in complex environments - understand ``non chain ladder
  like'').
\end{itemize}
\end{frame}

\begin{frame}
Those weights can be determined in different ways:

\begin{itemize}
\tightlist
\item
  Judgmentally, by considering the properties of the models available,
  and their respective strenghts and weaknesses for different \(i\).
\item
  With respect to some sort of objective criteria.

  \begin{itemize}
  \tightlist
  \item
    This is done to some extent in the book (Chapter 12).
  \item
    This has been done a lot more rigorously only very recently by
    \href{https://arxiv.org/abs/2206.08541}{Avanzi et al. (2023)} via
    ensembling, which was awarded the 2023 Hachemeister Prize by the
    American Casualty Actuarial Society (CAS).
  \item
    It is still a relevant topic!
  \end{itemize}
\end{itemize}
\end{frame}

\section{Allowance for prior
expectations}\label{allowance-for-prior-expectations}

\subsection{Idea}\label{idea-1}

\begin{frame}{Idea}
\begin{itemize}
\tightlist
\item
  Imagine one might to combine a ``prior expectation'' (or belief) with
  the estimate of liability provided by a method (or ensemble thereof).
\item
  This can be done in a way which is routinely referred to as
  ``credibility weighting'' by actuaries:
  \[ \text{estimate} = [1-z(i)] \overline{P_0^*}(i,k) + z(i) \overline{\hat{P}^*}(i,k),\]
  where

  \begin{itemize}
  \tightlist
  \item
    \(\overline{\hat{P}^*}(i,k)\) is the quantity defined earlier
  \item
    \(\overline{P_0^*}(i,k)\) is the prior expectation (examples later),
    and
  \item
    \(z(i)\) is the credibility assigned to the model estimate
    \(\overline{\hat{P}^*}(i,k)\)
  \end{itemize}
\item
  One probably should give more credibility to models in more mature
  years (small \(i\) ).
\item
  The formula above will yield equivalent results if \(z(i)\) is applied
  on incurred amounts instead (see Taylor (2000)).
\end{itemize}
\end{frame}

\subsection{Choice of credibility
weights}\label{choice-of-credibility-weights}

\begin{frame}{Bornhuetter-Ferguson}
\phantomsection\label{bornhuetter-ferguson}
Bornhuetter and Ferguson (1972) suggested the most simple approach,
which is to use either

\begin{itemize}
\tightlist
\item
  \(z(i)=0\): outstanding liability is exclusively calculated on the
  basis of prior expectations; or
\item
  \(z(i)=1\): outstanding liability is entirely based on models,
  ignoring prior expectations totally.
\end{itemize}

The ``textbook'', ``plain'' vanilla Bornhuetter-Ferguson (BF):

\begin{itemize}
\tightlist
\item
  applies \(z(i)\) on all or only some subset of the most immature years
\item
  calculates the prior expectation based on premium and loss ratios
\end{itemize}

An example is provided below.
\end{frame}

\begin{frame}{More generally}
\phantomsection\label{more-generally}
The following would make sense:

\begin{itemize}
\tightlist
\item
  \(z(i)=0\) when no information has been collected (start of
  development period 0);
\item
  \(z(i)=1\) at the end of the running off period (when all claims and
  their costs are known and certain);
\item
  some monotonic progression between those two extremes.
\end{itemize}

For instance, \(1/\pi\) from chain ladder:

\begin{itemize}
\tightlist
\item
  It satisfies the criteria above:
\item
  It is somewhat reflective of the amount of information gathered so far
\item
  In particular, a highly leveraged line, which would benefit from
  averaging with prior expectation, will have a very low
  \(z(I) = 1/\pi(0)\).
\end{itemize}
\end{frame}

\subsection{Prior expectations}\label{prior-expectations}

\begin{frame}{Prior expectations}
In this section we review some possible choices for the ``prior
expectations''.
\end{frame}

\begin{frame}{BF loss ratio}
\phantomsection\label{bf-loss-ratio}
The loss method proceeds as follows. For each \(i\)

\begin{itemize}
\tightlist
\item
  Define \(EP(i)\) as the gross aggregate premium earnt for period of
  origin \(i\).
\item
  Define \(C(i)\) as the aggregate sum of all payments made for period
  of origin \(i\) (the ultimate).
\item
  The loss ratio is then defined as \[ LR(i) = \frac{C(i)}{EP(i)}.\]
\end{itemize}

The method projects \(LR(i)\) from past values and infers \(C(i)\) from
observable \(EP(i)\). We have then (assuming chain ladder development
patterns)
\[\overline{P_0^*}(i,k) = LR(i)EP(i)\left( 1-\frac{1}{\pi(k)}\right)\]
which is typically applied with \(z(i)=1\) in immature (or all) years.
\end{frame}

\begin{frame}{Example}
\phantomsection\label{example}
Consider the following triangle (cumulative claims):

\begin{longtable}[]{@{}cccccc@{}}
\toprule\noalign{}
Origin & EP & DY1 & DY2 & DY3 & DY4 \\
\midrule\noalign{}
\endhead
2020 & 860 & 473 & 620 & 690 & 715 \\
2021 & 940 & 512 & 660 & 750 & \\
2022 & 980 & 611 & 700 & & \\
2023 & 1,020 & 647 & & & \\
\bottomrule\noalign{}
\end{longtable}

It is assumed that the ultimate loss ratios for underwriting years
2021-2023 are expected to be in line with year 2020.
\end{frame}

\begin{frame}
First calculate the development factors: \[\begin{aligned}
\widehat{f}_1 &= \frac{620+660+700}{473+512+611} = 1.2406 \\
\widehat{f}_2 &= \frac{690+750}{620+660} = 1.125 \\
\widehat{f}_3 &= \frac{715}{690} = 1.0362
\end{aligned}\] Then calculate the loss ratio \(LR\) and the prior
expectations of ultimate: \[\begin{aligned}
LR(1)EP(1) &=  LR \cdot 860 = 715 \Longrightarrow LR = 0.8314 \\
LR(2)EP(2)  &= LR \cdot 940 = 781.52\\
LR(3)EP(3)  &= LR \cdot 980 = 814.77\\
LR(4)EP(4)  &= LR \cdot 1020 = 848.02
\end{aligned}\]
\end{frame}

\begin{frame}
Now, we have

\[\begin{aligned}
\text{outstanding} &= \sum_{i=1}^4 \overline{P_0^*}(i,4) \\
&= \sum_{i=1}^4 LR(i)EP(i)\left(1-\frac{1}{\pi(i)}\right) \\
&= 0+27.30+115.82+261.64 \\
&=404.76.
\end{aligned}\]

Remember that \(\pi\) is normally defined as a \(\pi(j)\), so that
\(\pi(1)\) applies to that year where we have only \(j=1\) cell
available - the last row. Similarly, \(\pi(3)\) is the one that applies
to the amount in the diagonal that is at \(j=3\) - here 781.41.
\end{frame}

\begin{frame}{Extensions}
\phantomsection\label{extensions}
\begin{itemize}
\tightlist
\item
  The problem with the plain vanilla BF is that it does not specify
  objectively how \(LR(i)\) is determined. A priori, this is too
  judgmental.
\item
  There are a number of methods which try to make this choice more
  objective, such as ``modified BF'', ``Cape Cod'', \ldots{}
\end{itemize}
\end{frame}

\section*{References}\label{references}
\addcontentsline{toc}{section}{References}

\begin{frame}[allowframebreaks]{References}
\phantomsection\label{refs}
\begin{CSLReferences}{1}{0}
\bibitem[\citeproctext]{ref-AvLiWoXi23}
Avanzi, Benjamin, Yanfeng Li, Bernard Wong, and Alan Xian. 2023.
{``Ensemble Distributional Forecasting for Insurance Loss Reserving.''}

\bibitem[\citeproctext]{ref-BoFe72}
Bornhuetter, R., and R. Ferguson. 1972. {``The Actuary and
\textsc{IBNR}.''} \emph{Proceedings of The Casualty Actuarial Society}
59: 181--95.

\bibitem[\citeproctext]{ref-Tay00}
Taylor, Greg. 2000. \emph{Loss Reserving: An Actuarial Perspective}.
Huebner International Series on Risk, Insurance and Economic Security.
Kluwer Academic Publishers.

\end{CSLReferences}
\end{frame}

\end{document}
